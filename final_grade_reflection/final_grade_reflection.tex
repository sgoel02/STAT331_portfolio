% Options for packages loaded elsewhere
\PassOptionsToPackage{unicode}{hyperref}
\PassOptionsToPackage{hyphens}{url}
\PassOptionsToPackage{dvipsnames,svgnames,x11names}{xcolor}
%
\documentclass[
  letterpaper,
  DIV=11,
  numbers=noendperiod]{scrartcl}

\usepackage{amsmath,amssymb}
\usepackage{lmodern}
\usepackage{iftex}
\ifPDFTeX
  \usepackage[T1]{fontenc}
  \usepackage[utf8]{inputenc}
  \usepackage{textcomp} % provide euro and other symbols
\else % if luatex or xetex
  \usepackage{unicode-math}
  \defaultfontfeatures{Scale=MatchLowercase}
  \defaultfontfeatures[\rmfamily]{Ligatures=TeX,Scale=1}
\fi
% Use upquote if available, for straight quotes in verbatim environments
\IfFileExists{upquote.sty}{\usepackage{upquote}}{}
\IfFileExists{microtype.sty}{% use microtype if available
  \usepackage[]{microtype}
  \UseMicrotypeSet[protrusion]{basicmath} % disable protrusion for tt fonts
}{}
\makeatletter
\@ifundefined{KOMAClassName}{% if non-KOMA class
  \IfFileExists{parskip.sty}{%
    \usepackage{parskip}
  }{% else
    \setlength{\parindent}{0pt}
    \setlength{\parskip}{6pt plus 2pt minus 1pt}}
}{% if KOMA class
  \KOMAoptions{parskip=half}}
\makeatother
\usepackage{xcolor}
\setlength{\emergencystretch}{3em} % prevent overfull lines
\setcounter{secnumdepth}{-\maxdimen} % remove section numbering
% Make \paragraph and \subparagraph free-standing
\ifx\paragraph\undefined\else
  \let\oldparagraph\paragraph
  \renewcommand{\paragraph}[1]{\oldparagraph{#1}\mbox{}}
\fi
\ifx\subparagraph\undefined\else
  \let\oldsubparagraph\subparagraph
  \renewcommand{\subparagraph}[1]{\oldsubparagraph{#1}\mbox{}}
\fi


\providecommand{\tightlist}{%
  \setlength{\itemsep}{0pt}\setlength{\parskip}{0pt}}\usepackage{longtable,booktabs,array}
\usepackage{calc} % for calculating minipage widths
% Correct order of tables after \paragraph or \subparagraph
\usepackage{etoolbox}
\makeatletter
\patchcmd\longtable{\par}{\if@noskipsec\mbox{}\fi\par}{}{}
\makeatother
% Allow footnotes in longtable head/foot
\IfFileExists{footnotehyper.sty}{\usepackage{footnotehyper}}{\usepackage{footnote}}
\makesavenoteenv{longtable}
\usepackage{graphicx}
\makeatletter
\def\maxwidth{\ifdim\Gin@nat@width>\linewidth\linewidth\else\Gin@nat@width\fi}
\def\maxheight{\ifdim\Gin@nat@height>\textheight\textheight\else\Gin@nat@height\fi}
\makeatother
% Scale images if necessary, so that they will not overflow the page
% margins by default, and it is still possible to overwrite the defaults
% using explicit options in \includegraphics[width, height, ...]{}
\setkeys{Gin}{width=\maxwidth,height=\maxheight,keepaspectratio}
% Set default figure placement to htbp
\makeatletter
\def\fps@figure{htbp}
\makeatother

\KOMAoption{captions}{tableheading}
\makeatletter
\makeatother
\makeatletter
\makeatother
\makeatletter
\@ifpackageloaded{caption}{}{\usepackage{caption}}
\AtBeginDocument{%
\ifdefined\contentsname
  \renewcommand*\contentsname{Table of contents}
\else
  \newcommand\contentsname{Table of contents}
\fi
\ifdefined\listfigurename
  \renewcommand*\listfigurename{List of Figures}
\else
  \newcommand\listfigurename{List of Figures}
\fi
\ifdefined\listtablename
  \renewcommand*\listtablename{List of Tables}
\else
  \newcommand\listtablename{List of Tables}
\fi
\ifdefined\figurename
  \renewcommand*\figurename{Figure}
\else
  \newcommand\figurename{Figure}
\fi
\ifdefined\tablename
  \renewcommand*\tablename{Table}
\else
  \newcommand\tablename{Table}
\fi
}
\@ifpackageloaded{float}{}{\usepackage{float}}
\floatstyle{ruled}
\@ifundefined{c@chapter}{\newfloat{codelisting}{h}{lop}}{\newfloat{codelisting}{h}{lop}[chapter]}
\floatname{codelisting}{Listing}
\newcommand*\listoflistings{\listof{codelisting}{List of Listings}}
\makeatother
\makeatletter
\@ifpackageloaded{caption}{}{\usepackage{caption}}
\@ifpackageloaded{subcaption}{}{\usepackage{subcaption}}
\makeatother
\makeatletter
\@ifpackageloaded{tcolorbox}{}{\usepackage[many]{tcolorbox}}
\makeatother
\makeatletter
\@ifundefined{shadecolor}{\definecolor{shadecolor}{rgb}{.97, .97, .97}}
\makeatother
\makeatletter
\makeatother
\ifLuaTeX
  \usepackage{selnolig}  % disable illegal ligatures
\fi
\IfFileExists{bookmark.sty}{\usepackage{bookmark}}{\usepackage{hyperref}}
\IfFileExists{xurl.sty}{\usepackage{xurl}}{} % add URL line breaks if available
\urlstyle{same} % disable monospaced font for URLs
\hypersetup{
  pdftitle={Final Grade Reflection},
  pdfauthor={Sara Goel},
  colorlinks=true,
  linkcolor={blue},
  filecolor={Maroon},
  citecolor={Blue},
  urlcolor={Blue},
  pdfcreator={LaTeX via pandoc}}

\title{Final Grade Reflection}
\author{Sara Goel}
\date{}

\begin{document}
\maketitle
\ifdefined\Shaded\renewenvironment{Shaded}{\begin{tcolorbox}[borderline west={3pt}{0pt}{shadecolor}, boxrule=0pt, breakable, frame hidden, enhanced, sharp corners, interior hidden]}{\end{tcolorbox}}\fi

In this document, you make a data-based argument for the grade you've
earned in this course. Your argument should include evidence from the
supporting artifacts you've provided.

\hypertarget{the-output-document-should-be-a-pdf-or-a-word-document-as-it-should-be-a-maximum-of-2-pages.}{%
\subsection{\texorpdfstring{The output document should be a PDF or a
Word Document, as it should be a \textbf{maximum} of
2-pages.}{The output document should be a PDF or a Word Document, as it should be a maximum of 2-pages.}}\label{the-output-document-should-be-a-pdf-or-a-word-document-as-it-should-be-a-maximum-of-2-pages.}}

Learning Targets:

I believe I have made good progress in my proficiency of the learning
targets. I can import data from a variety of formats, as I demonstrated
in the week 2 preview activity (data\_import\_practice.qmd). I can
select different columns, filter through the rows and mutate or create
new variables in a dataset. I demonstrate this in Lab 3, questions 6 and
8. I have used semi, anti, left and right joins to combine dataframes,
as demonstrated in Lab 4, questions 5 and 6. My code is very tidy and
well documented, and I always make sure to include question headers,
chunk titles and white space. This can be seen in any lab assignment I
have submitted. I have learned to create visualizations for several data
types, and modified these visualizations to make sure they are clear and
easy to understand. One example where I have demonstrated proficiency in
this target is in Lab 5, question 3. I have learned how to write
efficient and concise code. I have also demonstrated proficiency in
using modern tools to do my data analysis. This can be seen in Challenge
4 in the chunk titled create-houses.

Continued Learning:

I am still working on creating professional looking analysis and am
continuing to make sure my code only outputs what the readers need to
see, rather than just my one debugging process. In this process I have
learned to find summaries of different variables and groups, as
summarizing all of the variables or groups in the output is not a final
solution to the problem, and it is important that the output contains
only the information that we are interested in knowing. I have showed
revision and continued learning in these topics in my Lab 4, question 4
in which I produced output that gives the one correct maximum value as
opposed to before, when I was printing the summaries of all of the
variables in descending order to find the maximum.

Peer Review / Teamwork:

I have been working very collaboratively with my team all quarter. I
often contribute to group discussions about the best way to solve a
problem, and together we evaluate which solutions are the most
efficient. I give helpful feedback in peer reviews for lab, making sure
to include both positive and constructive comments. This can be seen in
the screenshot of my comment for the lab 2 assigned peer review.

Final Grade:

I believe my final grade should be an A-. I am working hard to make sure
I understand the concepts we are learning and applying them to every
assignment, but I believe I could still be more mindful of the code I'm
writing to find the best possible solution, instead of just one that
works.



\end{document}
